\RequirePackage[l2tabu, orthodox]{nag}
\documentclass[11pt,paper=a4,oneside,ngerman,english,parskip=half]{scrartcl}

% !TeX spellcheck = en_GB

%\KOMAoptions{BCOR=6mm,DIV=12}
\KOMAoptions{DIV=12}

\usepackage{lmodern}
\renewcommand*\familydefault{\sfdefault}
\usepackage[T1]{fontenc}
\usepackage[utf8]{luainputenc}
% % %\setcounter{secnumdepth}{3}
% % %\setcounter{tocdepth}{3}
\usepackage{babel}
\usepackage[autostyle,english=british]{csquotes} % Anführungsstriche mit \enquote

\usepackage[style=phys,biblabel=brackets,backend=biber]{biblatex}
\addbibresource{C:/Users/gerald/Documents/library.bib}

% % % Tabellen, Listen, etc. % % %
% % %\usepackage{array}
\usepackage{longtable}
\usepackage{enumitem}		% customizable list environments
\setlist{noitemsep}
% % %\usepackage{float}
\usepackage{colortbl}
% Farben für die Tabelle
\definecolor{darkgrey}{rgb}{0.7,0.7,0.7}
\definecolor{lightgrey}{rgb}{0.85,0.85,0.85}

% % % % % % % Grafikpaket
\usepackage{graphicx, import}
%\usepackage[miktex]{gnuplottex}
%\usepackage{gnuplot-lua-tikz}
%\usepackage{tikz}
%\usetikzlibrary{babel,arrows.meta,intersections}
%\usepackage[margin=10pt,font=small,labelfont=bf,labelsep=colon]{caption}
%\usepackage[margin=20pt,font=small,labelfont=bf,labelsep=colon]{subcaption}
\usepackage{subfig}

% % % Mathematikpakete
\usepackage{multirow}
%\usepackage{amsmath}
\usepackage{mathtools}
\usepackage{amssymb}
\usepackage{amsthm}
\usepackage{esint} % für verschiedenste Integralzeichen
% \int \limts_{}^{} für über, unter dem Zeichen
\usepackage{xfrac} % für sfrac

\usepackage{makeidx}
\makeindex

\DeclareMathOperator{\sinc}{sinc}
%\usepackage{trsym} %für \TransformHoriz

% % % Physikalische Einheiten etc.
\usepackage{siunitx}
\sisetup{output-decimal-marker = {,}} % Dezimalzahlen in \num mit ,


\usepackage{microtype}

\usepackage[unicode=true,
	bookmarks=true,bookmarksnumbered=false,bookmarksopen=false,
	breaklinks=false,pdfborder={0 0 1},backref=false,colorlinks=false]
	{hyperref}
% % %\hypersetup{pdftitle={Akusto-optischer Effekt: Die Beugungsstärke in Abhängigkeit der Wellenfrontverkrümmung des Laserlichts},
% % %pdfauthor={Gerald Rapior},
% % % pdfkeywords={Akusto-optischer Effekt; Wellenfrontverkrümmung; Beugungsstärke}}

% % %\makeatletter


%\usepackage[OT2, OT1]{fontenc} % für russisch
%
%\newcommand\cyr{%
%\renewcommand\rmdefault{wncyr}%
%\renewcommand\sfdefault{wncyss}%
%\renewcommand\encodingdefault{OT2}%
%\normalfont
%\selectfont}
%\DeclareTextFontCommand{\textcyr}{\cyr}


\KOMAoption{captions}{topbeside,tableheading,nooneline}
\addtokomafont{captionlabel}{\bfseries}

\addto\captionsngerman{%
    \renewcommand{\figurename}{Abb.}%
    \renewcommand{\tablename}{Tab.}%
}

% % %\renewcommand*{\figureformat}{\figurename\,\thefigure}
% % %\renewcommand*{\tableformat}{\tablename\,\thetable}
% % %\renewcommand*{\captionformat}{.\ }
% % %\setcapindent{0pt}

\addto\captionsngerman{\renewcommand{\indexname}{Namen- und Sachverzeichnis}}
%\addto\captionsngerman{\renewcommand{\bibname}{Ergänzende und weiterführende Literatur}} % lässt sich nur in bibLatex umdefinieren


\usepackage{calc}
\renewcommand{\labelenumi}{\textbf{\arabic{chapter}.\theenumi.}}
\renewcommand{\labelenumii}{\textbf{\theenumii)}}
\setlist[enumerate,1]{align=left, leftmargin=0pt,labelwidth=*}
\setlist[enumerate,2]{align=left, leftmargin=0pt,labelwidth=*}


% Keine "Schusterjungen"
\clubpenalty = 10000
% Keine "Hurenkinder"
\widowpenalty = 10000
\displaywidowpenalty = 10000



\title{Volume of a simplex}
\author{Gerald Rapior}
\date{29.03.2016}
%\date{}

\begin{document}

\maketitle

\section{Definition of the problem}
\label{sec:DefProblem}

It should be shown that the volume of a simplex tends to 0 if the dimension of simplex grows.

Say $v_n(a)$ is the volume of a simplex where $n$ is the dimension. For convenience the volume will be determined for an arbitrary length $a$ and later $a = 1$ will be set. Than
\begin{equation}
\lim\limits_{n\rightarrow\infty} v(a = 1) = 0.
\end{equation}

In geometry, a simplex (plural: simplexes or simplices) is a generalization of the notion of a triangle or tetrahedron to arbitrary dimensions (see \href{https://en.wikipedia.org/wiki/Simplex}{Simplex in Wikipedia}).
For simplicity a canonical basis of unit vectors $\vec{e_i}$ is used. The simplex is determined by a set of points:
\begin{equation}
C = \left\{ c_1 \vec{e_1} + \cdots + c_n \vec{e_n}\,| \,\forall i,\,c_i\geq 0,\, \sum_{i=1}^{n} c_i = 1 \right\}.
\end{equation}
Borders of such a simplex are the unit vectors and straight lines between end points of the unit vectors respectively.


\section{Determination of volume}
\label{sec:Volume}

\begin{itemize}
\item $n = 1$:
\begin{figure}
\centering
\import{./pics/}{1-dim.pdf_tex}
\caption{Volume of a 1-D simplex}
\label{fig.1-dim}
\end{figure}

In figure \ref{fig.1-dim} is shown a 1-dimensional simplex (a straight line). The length (volume in one dimension) is determined by:
\begin{equation}
v_1(a) = \int\limits_{0}^{a}\text{d}x =\biggl[x\biggl]_{0}^{a} = a.
\label{eq:1-dim-vol}
\end{equation}
Later the $a$ can be substituted by 1.

\item $n = 2$:
\begin{figure}
\centering
\import{./pics/}{2-dim.pdf_tex}
\caption{Volume of a 2-D simplex}
\label{fig.2-dim}
\end{figure}

In figure \ref{fig.2-dim} is shown a 2-dimensional simplex (a triangle). The surface of the shaded triangle is determined by:
\begin{equation}
v_2(a) = \int\limits_{0}^{a} \int\limits_{0}^{y} \text{d}x\,\text{d}y  = \int\limits_{0}^{a} y \,\text{d}y = \left[\frac{1}{2}\,y^2\right]_{0}^{a} = \frac{a^2}{2}.
\end{equation}

\item $n = 3$:
\begin{figure}
\centering
\import{./pics/}{3-dim.pdf_tex}
\caption{Volume of a 3-D simplex}
\label{fig.3-dim}
\end{figure}

The volume of the shaded area (see fig.\,\ref{fig.3-dim}) is determined by \footnote{The dashed triangle in fig.\,\ref{fig.3-dim} is an arbitrary similar triangle of the \enquote{base}. The size is direct related to the position $z$. At least all these triangles are \enquote{added together}. This shows clear the recursive way to construct volumes of higher dimensions.}:

\begin{equation}
v_3(a) = \int\limits_{0}^{a} \int\limits_{0}^{z} \int\limits_{0}^{y}\text{d}x\,\text{d}y\,\text{d}z  = \int\limits_{0}^{a} \int\limits_{0}^{z} y \,\text{d}y\,\text{d}z = \int\limits_{0}^{a} \frac{z^2}{2} \,\text{d}z =\left[\frac{1}{2\cdot 3}\,z^3\right]_{0}^{a} = \frac{a^3}{6}.
\end{equation}

\end{itemize}

It seems that the volume $v_n(a)$ of dimension $n$ for $a = 1$ is:
\begin{equation}
\label{eq:volume}
v_n(1) = \frac{1}{n!}.
\end{equation}


\section{Proof of (\ref{eq:volume})}
The proof is made by means of induction.
$v_n(a) = \sfrac{1}{n!}$ for $n = 1$ is shown above \eqref{eq:1-dim-vol}. Say the equation is valid for $n$. Next it has to be shown that the equation is valid for $n+1$.
\begin{proof}
\begin{align}
v_{n+1}(a) &= \int\limits_{0}^{a} \underbrace{\int\limits_{0}^{x_{n+1}} \cdots \int\limits_{0}^{x_2} \text{d}x_1 \cdots \text{d}x_n}_{=\;v_n(x_{n+1})}  \,\text{d}x_{n+1} \\
 &= \int\limits_{0}^{a} \frac{1}{n!} \, x_{n+1}^n \,\text{d}x_{n+1} \\
 &= \left[\frac{1}{n!}\,\frac{1}{n+1}\,x_{n+1}^{n+1} \right]_0^{a} = \frac{a^{n+1}}{(n+1)!} \qedhere
\end{align}
\end{proof}

In the formula above old $x$, $y$ and $z$ are replaced by $x_1$, $x_2$ and $x_3$. For $a = 1$ we get result for volume of a simplex.

At least we know that
\begin{equation}
\lim\limits_{n\rightarrow \infty}v_n(1) = \lim\limits_{n\rightarrow \infty} \frac{1}{n!} = 0.
\end{equation}


\printbibliography[title=Weiterführende Literatur]
\end{document}
