\RequirePackage[l2tabu, orthodox]{nag}
\documentclass[11pt,paper=a4,oneside,ngerman,english,parskip=half]{scrartcl}

% !TeX spellcheck = en_GB

%\KOMAoptions{BCOR=6mm,DIV=12}
\KOMAoptions{DIV=12}

\usepackage{lmodern}
\usepackage[T1]{fontenc}
\usepackage[utf8]{luainputenc}
% % %\setcounter{secnumdepth}{3}
% % %\setcounter{tocdepth}{3}
\usepackage{babel}
\usepackage[autostyle,english=british]{csquotes} % Anführungsstriche mit \enquote

\usepackage[style=phys,biblabel=brackets,backend=biber]{biblatex}
\addbibresource{C:/Users/gerald/Documents/library.bib}

% % % Tabellen, Listen, etc. % % %
% % %\usepackage{array}
\usepackage{longtable}
\usepackage{enumitem}		% customizable list environments
\setlist{noitemsep}
% % %\usepackage{float}
\usepackage{colortbl}
% Farben für die Tabelle
\definecolor{darkgrey}{rgb}{0.7,0.7,0.7}
\definecolor{lightgrey}{rgb}{0.85,0.85,0.85}

% % % % % % % Grafikpaket
\usepackage{graphicx, import}
%\usepackage[miktex]{gnuplottex}
%\usepackage{gnuplot-lua-tikz}
%\usepackage{tikz}
%\usetikzlibrary{babel,arrows.meta,intersections}
%\usepackage[margin=10pt,font=small,labelfont=bf,labelsep=colon]{caption}
%\usepackage[margin=20pt,font=small,labelfont=bf,labelsep=colon]{subcaption}
\usepackage{subfig}

% % % Mathematikpakete
\usepackage{multirow}
%\usepackage{amsmath}
\usepackage{mathtools}
\usepackage{amssymb}
\usepackage{esint} % für verschiedenste Integralzeichen
% \int \limts_{}^{} für über, unter dem Zeichen
\usepackage{xfrac} % für sfrac

\usepackage{makeidx}
\makeindex

\DeclareMathOperator{\sinc}{sinc}
%\usepackage{trsym} %für \TransformHoriz

% % % Physikalische Einheiten etc.
\usepackage{siunitx}
\sisetup{output-decimal-marker = {,}} % Dezimalzahlen in \num mit ,


\usepackage{microtype}

\usepackage[unicode=true,
	bookmarks=true,bookmarksnumbered=false,bookmarksopen=false,
	breaklinks=false,pdfborder={0 0 1},backref=false,colorlinks=false]
	{hyperref}
% % %\hypersetup{pdftitle={Akusto-optischer Effekt: Die Beugungsstärke in Abhängigkeit der Wellenfrontverkrümmung des Laserlichts},
% % % pdfauthor={Gerald Rapior},
% % % pdfkeywords={Akusto-optischer Effekt; Wellenfrontverkrümmung; Beugungsstärke}}

% % %\makeatletter


%\usepackage[OT2, OT1]{fontenc} % für russisch
%
%\newcommand\cyr{%
%\renewcommand\rmdefault{wncyr}%
%\renewcommand\sfdefault{wncyss}%
%\renewcommand\encodingdefault{OT2}%
%\normalfont
%\selectfont}
%\DeclareTextFontCommand{\textcyr}{\cyr}


\KOMAoption{captions}{topbeside,tableheading,nooneline}
\addtokomafont{captionlabel}{\bfseries}

\addto\captionsngerman{%
    \renewcommand{\figurename}{Abb.}%
    \renewcommand{\tablename}{Tab.}%
}

% % %\renewcommand*{\figureformat}{\figurename\,\thefigure}
% % %\renewcommand*{\tableformat}{\tablename\,\thetable}
% % %\renewcommand*{\captionformat}{.\ }
% % %\setcapindent{0pt}

\addto\captionsngerman{\renewcommand{\indexname}{Namen- und Sachverzeichnis}}
%\addto\captionsngerman{\renewcommand{\bibname}{Ergänzende und weiterführende Literatur}} % lässt sich nur in bibLatex umdefinieren


\usepackage{calc}
\renewcommand{\labelenumi}{\textbf{\arabic{chapter}.\theenumi.}}
\renewcommand{\labelenumii}{\textbf{\theenumii)}}
\setlist[enumerate,1]{align=left, leftmargin=0pt,labelwidth=*}
\setlist[enumerate,2]{align=left, leftmargin=0pt,labelwidth=*}


% Keine "Schusterjungen"
\clubpenalty = 10000
% Keine "Hurenkinder"
\widowpenalty = 10000
\displaywidowpenalty = 10000



\title{Volume of a simplex}
\author{Gerald Rapior}
\date{29.03.2016}
%\date{}

\begin{document}

\maketitle

\section{Definition of the problem}
\label{sec:DefProblem}

It should be shown that the volume of a simplex tends to 0 if the dimension of simplex grows.

Say $n$ is the dimension and $v$ the volume of a simplex than
\begin{equation}
\lim_{n\rightarrow\infty} v(n) = 0.
\end{equation}

In geometry, a simplex (plural: simplexes or simplices) is a generalization of the notion of a triangle or tetrahedron to arbitrary dimensions (see \href{https://en.wikipedia.org/wiki/Simplex}{Simplex in Wikipedia}).
For simplicity a canonical basis of unit vectors $\vec{e_i}$ is used. The simplex is determined by the set of points:
\begin{equation}
C = \left\{ c_1 \vec{e_1} + \cdots + c_n \vec{e_n}\,|\,c_i\geq 0, 0 \leq i \leq n, \sum_{i=1}^{n} c_i = 1 \right\}.
\end{equation}
The borders of such a simplex are the unit vectors and straight lines between the end points of the unit vectors respectively.


\section{Determination of volume}

$n = 1$:
Bild Linie

The length (volume in one dimension) is determined by:
\begin{equation}
v(1) = \int\limits_{0}^{a}\text{d}x =\biggl[x\biggl]_{0}^{a} = a.
\end{equation}
Later the $a$ can be substitute by 1.

$n = 2$:
Bild Dreieck

The surface of the shaded triangle is determined by:
\begin{equation*}
v(2) = \int\limits_{0}^{a} \int\limits_{0}^{y} \text{d}x\,\text{d}y  = \int\limits_{0}^{a} y \,\text{d}y = \left[\frac{1}{2}y^2\right]_{0}^{a} = \frac{a^2}{2}.
\end{equation*}

$n = 3$:
Bild Wuerfel

The volume of the shaded area is determined by:

\begin{equation*}
v(3) = \int\limits_{0}^{a} \int\limits_{0}^{z} \int\limits_{0}^{y}\text{d}x\,\text{d}y\,\text{d}z  = \int\limits_{0}^{a} \int\limits_{0}^{z} y \,\text{d}y\,\text{d}z = \int\limits_{0}^{a} \frac{z^2}{2} \,\text{d}z =\left[\frac{1}{2\cdot 3}z^3\right]_{0}^{a} = \frac{a^3}{6}.
\end{equation*}

It seems that the volume $v(n)$ of dimension n for $a = 1$ is:
\begin{equation}
v(n) = \frac{1}{n!}.
\end{equation}


\section{Proof of the volume}
The proof is made by induction.
$v(n) = \sfrac{1}{n!}$ for $n = 1$ is shown above. Say the equation is valid for $n$. Next it has to be shown that the equation is valid for $n+1$. 







%\begin{figure}
%\centering
%\input{./pics/Erg-Feld-b20v0_8t0_04.tex}
%\caption{Feldverteilung bei $b=\num{20}$ und $v=\num{0.8}$}
%\label{fig.Erg-Feld-b20v0_8t0_04}
%\end{figure}





\printbibliography[title=Weiterführende Literatur]
\end{document}
